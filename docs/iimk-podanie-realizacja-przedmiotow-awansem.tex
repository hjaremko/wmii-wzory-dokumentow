\PassOptionsToPackage{pdftitle={Realizacja przedmiotow awansem}}{hyperref}

\documentclass{wmiisubmission}

\usepackage{tabularx}

\geometry{total={210mm,297mm},
left=25mm,right=25mm,%
bindingoffset=0mm, top=15mm,bottom=10mm}

\linespread{1.1}

\renewcommand{\thefootnote}{\fnsymbol{footnote}}

\newcommand{\fnt}[1]{
    \addtocounter{footnote}{-\value{footnotemarknum}}
    \addtocounter{footnote}{1}
    \footnotetext{#1}
    \setcounter{footnotemarknum}{0}
}

\begin{document}
\cracowdate
\studentinfo{}{}{}{Informatyka/Matematyka Komputerowa*}{I stopnia/II stopnia*}
\studentaddress
\addressee{\piotrniemiec}

\vskip 1.6cm

Oświadczam, że w roku akademickim \fillField{1.2cm}/\fillField{1.2cm}~ będę realizował/ła
\textbf{awansem} niżej
wymienione przedmioty:\\

Semestr: zimowy / letni\footnote[1]{niepotrzebne skreślić}\\\\

\begin{tabularx}{\textwidth}{|l|X|l|l|l|}

    \hline
    \textbf{Lp.} & \textbf{Nazwa przedmiotu} \hspace{0.9cm} & \small{\textbf{Liczba godzin}} & \textbf{ECTS} & \footnotesize{\textbf{Przedmiot obowiązkowy z roku}} \\
    \hline
    1.  &   &   &  &\\
        &   &   &  &\\
    \hline
    2.  &   &   &  &\\
        &   &   &  &\\
    \hline
    3.  &   &   &  &\\
        &   &   &  &\\
    \hline
    4.  &   &   &  &\\
        &   &   &  &\\
    \hline
    5.  &   &   &  &\\
        &   &   &  &\\
    \hline

\end{tabularx}

\vskip 1.4cm

Przyjmuję do wiadomości, że powyżej wymienione przedmioty i egzaminy są dla mnie
obowiązkowe w programie studiów tego roku i niezaliczenie lub niezdanie egzaminu
pociągnie konsekwencje zgodne z regulaminem studiów (konieczność płatnego
powtarzania przedmiotu).

\vskip 1.6cm

\studentsignature

\vskip 1.5cm

\decision{Decyzja}

\end{document}
