\PassOptionsToPackage{pdftitle={Wniosek o uzyskanie wpisu na kolejny rok studiow}}{hyperref}

\documentclass{wmiisubmission}

\usepackage{tabularx}

\geometry{total={210mm,297mm},
left=25mm,right=25mm,%
bindingoffset=0mm, top=15mm,bottom=10mm}

\linespread{1.1}

\begin{document}
\cracowdate
\studentinfo{}{}{}{@COURSE@ - studia stacjonarne}{I stopnia}
\studentaddress
\addressee[-3em]{\piotrniemiec}
\vskip 1.0cm

\begin{center}
{\Large \textbf{Wniosek o uzyskanie wpisu na kolejny rok studiów}}
\end{center}

\vskip 0.5cm

W związku z uzyskaniem przeze mnie \fillField{2cm}~ pkt. (min. 50 pkt.) na roku
\dotfill\\ uprzejmie proszę o \textbf{wpisanie mnie na rok}
\fillField{3cm}, z obowiązkiem uzupełnienia różnicy punktowej z n/w przedmiotu/ów
 w wysokości \fillField{2cm} pkt., do 30 września 20\fillField{1cm} r.\\

\begin{tabularx}{\textwidth}{|l|X|l|l|l|}
    \hline
    \textbf{Lp.} & \textbf{Nazwa przedmiotu} \hspace{0.9cm} & {\textbf{Semestr}} & \textbf{ECTS} & {\textbf{Liczba godzin}} \\
    \hline
    1.  &   &   &  &\\
        &   &   &  &\\
    \hline
    2.  &   &   &  &\\
        &   &   &  &\\
    \hline
    3.  &   &   &  &\\
        &   &   &  &\\
    \hline
\end{tabularx}

\vskip 0.3cm

\begin{minipage}{\textwidth}
    \setstretch{0.5}
    \footnotesize
    \bf
    Opłata za 1 godzinę jest zgodna z podpisaną umową i wynosi odpowiednio dla studentów rozpoczynających studia w roku akademickim 2013/14 i 14/15 – 9,00 zł, 15/16 – 8,00 zł; 16/17 – 7,00zł
\end{minipage}

\vfill

\underline{Kwota do zapłaty:}\\
\textit{Semestr zimowy [wpłata do 22.10]}\\
Liczba punktów/godzin \dotfill $\times$ kwota \dotfill \\\\
\textit{Semestr letni [wpłata do 15.03]}\\
Liczba punktów/godzin \dotfill $\times$ kwota \dotfill \\

\vskip 0.6cm
\hspace{\fill} Razem: \fillField{6cm} \hspace{2.0cm}

\vskip 1.6cm
\studentsignature
\vfill

\decision{Decyzja Kierownika}

\end{document}
