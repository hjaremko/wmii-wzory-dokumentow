\PassOptionsToPackage{pdftitle={Wniosek o zwolnienie z oplat}}{hyperref}

\documentclass{wmiisubmission}
\usepackage{enumitem,amssymb}

\geometry{total={210mm,297mm},
left=25mm,right=25mm,%
bindingoffset=0mm, top=15mm,bottom=10mm}

\linespread{1.1}

\renewcommand{\thefootnote}{\fnsymbol{footnote}}

\newlist{todolist}{itemize}{2}
\setlist[todolist]{label=$\square$}

\begin{document}
\cracowdate
\studentinfo{}{}{}{@COURSE@ - studia stacjonarne}{I stopnia/II stopnia}
\studentaddress
\vskip 0.8cm
\fillField{8cm}[(właściwy Urząd Skarbowy)]

\addressee[-1.5cm]{\piotrniemiec}

Wniosek o zwolnienie z opłat za usługi edukacyjne – powtarzanie przedmiotów\\
(kwota: \fillField{4cm} ~~PLN).\\

Na podstawie:
\begin{todolist}
    \itemsep0em
    \item Istotnego pogorszenia sytuacji materialnej w rodzinie
    \item Uzyskanie wybitnych wyników w nauce
    \item Wypadku losowego
    \item Inny \dotfill
\end{todolist}

Uzasadnienie:

\phantom{a}\dotfill

\phantom{a}\dotfill

\vskip 0.6cm
\studentsignature
\vskip 0.6cm

{\footnotesize Oświadczam, iż jestem świadomy/a, że całkowite lub częściowe
zwolnienie z opłaty jest to dochód, od którego nie zostały pobrane zaliczki na
podatek dochodowy,
\textbf{a obowiązek odprowadzenia podatku do fiskusa spoczywa na barkach studenta}.
Przychody te wykazane są na formularzu PIT-8C w części D. Formularz PIT-8C,
zostanie przygotowany przez Uczelnie w ustawowym terminie}.

\vskip 0.8cm
\studentsignature
\vskip 0.8cm

Dochód na członka rodziny {\tiny[potwierdzony przez Koordynatora pomocy materialnej]}: \dotfill

\vfill

\decision{Opinia Kierownika}

\decision{Decyzja Dziekana}

\end{document}
