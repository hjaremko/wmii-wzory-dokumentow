\PassOptionsToPackage{pdftitle={Powtarzanie przedmiotu}}{hyperref}

\documentclass{wmiisubmission}
\usepackage{tabularx}

\geometry{total={210mm,297mm},
left=25mm,right=25mm,%
bindingoffset=0mm, top=15mm,bottom=15mm}

\newcounter{footnotemarknum}
\newcommand{\fnm}{\addtocounter{footnotemarknum}{1}\footnotemark}

\newcommand{\fnt}[1]{
    \addtocounter{footnote}{-\value{footnotemarknum}}
    \addtocounter{footnote}{1}
    \footnotetext{#1}
    \setcounter{footnotemarknum}{0}
}

\begin{document}
\cracowdate
\studentinfo{}{}{}{Matematyka Komputerowa - studia stacjonarne}{I stopnia/II stopnia*}
\studentaddress
\addressee[-1em]{\piotrniemiec}

\vskip 1.5cm

Proszę o naliczenie należności za powtarzanie n/w przedmiotu(ów) w roku
akademickim \fillField{1cm}/\fillField{1cm}.\\

\begin{tabularx}{\textwidth}{|l|X|l|l|l|}

    \hline
    \textbf{Lp.} & \textbf{Nazwa przedmiotu} \hspace{0.9cm} & {\textbf{Semestr}} & \textbf{ECTS} & {\textbf{Liczba godzin}} \\
    \hline
    1.  &   &   &  &\\
        &   &   &  &\\
    \hline
    2.  &   &   &  &\\
        &   &   &  &\\
    \hline
    3.  &   &   &  &\\
        &   &   &  &\\
    \hline
    4.  &   &   &  &\\
        &   &   &  &\\
    \hline

\end{tabularx}

\vskip 0.3cm

\begin{minipage}{\textwidth}
    \setstretch{0.5}
    \footnotesize
    \bf
    Opłata za 1 godzinę jest zgodna z podpisaną umową i wynosi odpowiednio dla studentów rozpoczynających studia w roku akademickim 2013/14 i 14/15 – 9,00 zł, 15/16 – 8,00 zł; 16/17 – 7,00zł
\end{minipage}

\vfill

\underline{Kwota do zapłaty:}\\
\textit{Semestr zimowy [wpłata do 22.10]}\\
Liczba punktów/godzin \dotfill $\times$ kwota \dotfill \\\\
\textit{Semestr letni [wpłata do 15.03]}\\
Liczba punktów/godzin \dotfill $\times$ kwota \dotfill \\

\hspace{\fill} Razem: \fillField{6cm} \hspace{2.0cm}

\vskip 1.6cm
\studentsignature
\vfill

\decision{Decyzja Kierownika}

\end{document}
